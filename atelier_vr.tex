\documentclass[12pt]{article}
\usepackage[french]{babel}
\usepackage[utf8]{inputenc}
\usepackage[T1]{fontenc}
\usepackage{hyperref}
\usepackage{xcolor}
\usepackage{minted}
\usemintedstyle{friendly}
% Default minted settings for code blocks: encadrer les blocs et fond léger
\definecolor{codebg}{RGB}{250,250,250}
\definecolor{coderule}{RGB}{200,200,200}
% Options: frame=single ajoute une bordure, bgcolor définit le fond, rulecolor pour la bordure
\setminted{fontsize=\small,breaklines=true,linenos=false,autogobble=true,encoding=utf8,frame=single,bgcolor=codebg,rulecolor=coderule}
\usepackage{geometry}
\geometry{a4paper, margin=1in}

% Using minted for syntax highlighting instead of listings

\title{Atelier Découverte de la VR Interactive}
\author{Théo AVRIL}
\date{\today}

\begin{document}
\maketitle

\section{Objectifs}
Ce tutoriel vous permettra de :
\begin{itemize}
    \item Comprendre la structure de base d'un document HTML.
    \item Découvrir la bibliothèque AFrame pour créer une scène VR.
    \item Ajouter une interactivité simple en JavaScript (changer la couleur d'un objet au clic).
\end{itemize}

\section{Matériel Nécessaire}
\begin{itemize}
    \item Un ordinateur avec un navigateur moderne (Chrome, Firefox, etc.).
    \item Un éditeur de texte (Visual Studio Code, Sublime Text, etc.).
    \item Optionnel : un casque VR ou un smartphone compatible pour le mode VR.
\end{itemize}

\section{Étape 1 : Préparer votre Environnement}
\subsection{Création d'un fichier HTML}
\begin{enumerate}
    \item Ouvrez votre éditeur de texte et créez un nouveau fichier nommé \texttt{index.html}.
\end{enumerate}

\subsection{Structure de Base}
Copiez-collez le code suivant dans votre fichier \texttt{index.html} :

\begin{minted}{html}
<!DOCTYPE html>
<html>
  <head>
    <meta charset="utf-8">
    <title>Atelier VR Interactif</title>
    <!-- Inclusion de la bibliothèque AFrame -->
    <script src="https://aframe.io/releases/1.2.0/aframe.min.js"></script>
  </head>
  <body>
    <!-- La scène VR sera définie ici -->
  </body>
</html>
\end{minted}

\section{Étape 2 : Créer une Scène VR de Base}
\subsection{Ajouter la Scène et des Objets}
Dans la balise \texttt{<body>}, ajoutez la balise \texttt{<a-scene>} avec le contenu suivant :

\begin{minted}{html}
<body>
  <a-scene>
    <!-- Cube interactif : clic pour changer de couleur -->
    <a-box position="-1 0.5 -3" rotation="0 45 0" color="#4CC3D9"
           animation="property: rotation; to: 0 405 0; loop: true; dur: 4000"
           change-color-on-click></a-box>
    
    <!-- Une sphère -->
    <a-sphere position="0 1.25 -5" radius="1.25" color="#EF2D5E"></a-sphere>
    
    <!-- Un cylindre -->
    <a-cylinder position="1 0.75 -3" radius="0.5" height="1.5" color="#FFC65D"></a-cylinder>
    
    <!-- Un sol -->
    <a-plane position="0 0 -4" rotation="-90 0 0" width="4" height="4" color="#7BC8A4"></a-plane>
    
    <!-- Un ciel -->
    <a-sky color="#ECECEC"></a-sky>
  </a-scene>
</body>
\end{minted}

\subsection{Explications}
\begin{itemize}
    \item \textbf{\texttt{<a-scene>}} : Crée l'environnement VR.
    \item \textbf{\texttt{<a-box>, <a-sphere>, <a-cylinder>, <a-plane>, <a-sky>}} : Représentent respectivement un cube, une sphère, un cylindre, un sol et un ciel.
    \item L'attribut \texttt{animation} sur le cube fait tourner l'objet en continu.
    \item L'attribut \texttt{change-color-on-click} sera utilisé pour activer l'interactivité (voir l'étape suivante).
\end{itemize}

\section{Étape 3 : Ajouter l'Interactivité avec JavaScript}
\subsection{Création d'un Composant Personnalisé}
Nous allons créer un composant AFrame nommé \texttt{change-color-on-click} qui change la couleur d'un objet lorsqu'on clique dessus.

\subsection{Ajouter le Script JavaScript}
Ajoutez le code suivant juste avant la fermeture de la balise \texttt{</body>} :

\begin{minted}{html}
<!-- Script JavaScript pour ajouter l'interactivité -->
<script>
  // Enregistrement d'un composant personnalisé "change-color-on-click"
  AFRAME.registerComponent('change-color-on-click', {
    init: function () {
      // Ecoute de l'événement 'click' sur l'élément
      this.el.addEventListener('click', function () {
        // Génère une couleur aléatoire au format hexadécimal
        var randomColor = '#' + Math.floor(Math.random() * 16777215).toString(16);
        // Change la couleur de l'élément cliqué
        this.setAttribute('color', randomColor);
      });
    }
  });
</script>
\end{minted}

\subsection{Explications}
\begin{itemize}
    \item \textbf{AFRAME.registerComponent} : Permet de créer un composant personnalisé dans AFrame.
    \item La fonction \texttt{init} est exécutée lors de l'initialisation de l'élément.
    \item \texttt{addEventListener('click', ...)} : Ajoute un écouteur d'événement pour détecter le clic sur l'élément.
    \item Le script génère une couleur hexadécimale aléatoire et met à jour l'attribut \texttt{color} de l'élément cliqué.
\end{itemize}

\section{Étape 4 : Tester et Personnaliser}
\subsection{Tester la Scène VR}
\begin{enumerate}
    \item Sauvegardez votre fichier \texttt{index.html}.
    \item Ouvrez-le dans un navigateur moderne. Vous verrez la scène VR avec le cube, la sphère, le cylindre, le sol et le ciel.
    \item Cliquez sur le cube pour observer le changement de couleur.
\end{enumerate}

\subsection{Explorer et Modifier}
\begin{itemize}
    \item Modifiez les valeurs des attributs \texttt{position}, \texttt{color} ou \texttt{rotation} pour observer les changements.
    \item Ajoutez de nouveaux objets, par exemple en utilisant les balises \texttt{<a-cone>} ou \texttt{<a-torus>}.
    \item Expérimentez en modifiant l'interactivité, comme changer la taille ou la position de l'objet lors d'un clic.
\end{itemize}

\section{Glossaire et Documentation A-Frame}

Cette section vous propose une documentation rapide des commandes et attributs les plus utiles d'A-Frame pour créer vos propres scènes VR.

\subsection{Formes et Objets 3D de Base}

\begin{minted}{html}
<!-- Formes primitives -->
<a-box>          <!-- Un cube ou parallelepipede -->
<a-sphere>       <!-- Une sphere -->
<a-cylinder>     <!-- Un cylindre -->
<a-plane>        <!-- Un plan (surface plate) -->
<a-cone>         <!-- Un cone -->
<a-torus>        <!-- Un tore (forme d'anneau/donut) -->
<a-ring>         <!-- Un anneau plat (cercle avec trou) -->
<a-circle>       <!-- Un cercle plat -->
<a-triangle>     <!-- Un triangle plat -->
<a-sky>          <!-- Un ciel spherique (environnement) -->
\end{minted}

\subsection{Attributs Communs}

Ces attributs peuvent être appliqués à la plupart des objets :

\begin{minted}{html}
<!-- Position, Rotation, et Echelle -->
position="x y z"     <!-- Position en metres (x: gauche/droite, y: haut/bas, z: avant/arriere) -->
rotation="x y z"     <!-- Rotation en degres (autour des axes x, y, z) -->
scale="x y z"        <!-- Echelle (1 1 1 par defaut = taille normale) -->

<!-- Apparence -->
color="#ff0000"      <!-- Couleur (format hexadecimal) -->
opacity="0.5"        <!-- Opacite (0 = transparent, 1 = opaque) -->
visible="true/false" <!-- Visibilite de l'objet -->
metalness="0.5"      <!-- Aspect metallique (0 = mat, 1 = metallique) -->
roughness="0.5"      <!-- Rugosite (0 = lisse/reflechissant, 1 = rugueux) -->

<!-- Dimensions specifiques -->
width="2"            <!-- Largeur (pour box, plane, etc.) -->
height="1"           <!-- Hauteur (pour box, plane, etc.) -->
depth="0.5"          <!-- Profondeur (pour box) -->
radius="1.5"         <!-- Rayon (pour sphere, cylinder, etc.) -->
\end{minted}

\subsection{Images et Textures}

\begin{minted}{html}
<!-- Appliquer une image comme texture -->
<a-box src="chemin/vers/image.jpg"></a-box>

<!-- Image comme arriere-plan du ciel -->
<a-sky src="ciel.jpg"></a-sky>

<!-- Images 360 degres pour environnement -->
<a-sky src="panorama360.jpg"></a-sky>

<!-- Image sur un plan (comme un poster) -->
<a-image src="image.png" position="0 1.5 -2" width="2" height="1"></a-image>
\end{minted}

\subsection{Lumières}

\begin{minted}{html}
<!-- Types de lumieres -->
<a-light type="ambient" color="#BBB"></a-light>       <!-- Eclairage ambiant -->
<a-light type="directional" position="0 1 1"></a-light>  <!-- Lumiere directionnelle -->
<a-light type="point" position="0 2 0"></a-light>     <!-- Point lumineux -->
<a-light type="spot" position="0 2 0" angle="45"></a-light> <!-- Projecteur -->

<!-- Attributs pour lumieres -->
intensity="0.75"     <!-- Intensite de la lumiere (0-1) -->
distance="50"        <!-- Distance d'attenuation -->
angle="60"           <!-- Angle du faisceau pour spot (degres) -->
penumbra="0.2"       <!-- Flou au bord pour spot (0-1) -->
\end{minted}

\subsection{Animation et Interactivité}

\begin{minted}{html}
<!-- Animation simple -->
<a-box position="0 1 -3" animation="property: rotation; to: 0 360 0; loop: true; dur: 2000"></a-box>

<!-- Attributs d'animation -->
animation="property: position; to: 0 2 -3; dur: 2000; easing: easeInOutQuad; loop: true"

<!-- Proprietes animables principales -->
property="position"    <!-- Propriete a animer (position, rotation, scale, color...) -->
from="0 1 -3"         <!-- Valeur de depart (optionnel) -->
to="0 2 -3"           <!-- Valeur d'arrivee -->
dur="2000"            <!-- Duree en millisecondes -->
delay="500"           <!-- Delai avant demarrage -->
loop="true/false"     <!-- Animation en boucle -->
dir="alternate"       <!-- Direction (normal, reverse, alternate) -->
easing="easeInOutQuad" <!-- Type d'acceleration -->
\end{minted}

\subsection{Événements et Interactions}

\begin{minted}{html}
<!-- Curseur pour interactivite -->
<a-camera>
  <a-cursor color="#FFFFFF"></a-cursor>
</a-camera>

<!-- Composants d'interaction predefinis -->
<a-entity cursor="rayOrigin: mouse"></a-entity>   <!-- Curseur suivant la souris -->

<!-- Declaration simplifiee d'un script pour gerer un evenement -->
<script>
  // Exemple : changer la couleur au clic
  document.querySelector('#monElement').addEventListener('click', function() {
    this.setAttribute('color', 'red');
  });
  
  // Evenements disponibles
  // click        - au clic ou tap
  // mouseenter   - quand le curseur passe sur l'objet
  // mouseleave   - quand le curseur quitte l'objet
  // mousedown    - quand on appuie (sans relacher)
  // mouseup      - quand on relache le clic
</script>
\end{minted}

\subsection{Chargement de Modèles 3D}

\begin{minted}{html}
<!-- Chargement de modeles 3D -->
<a-entity gltf-model="url(modele.glb)" scale="0.5 0.5 0.5" position="0 0 -5"></a-entity>

<!-- Formats supportes -->
<a-entity obj-model="obj: url(modele.obj); mtl: url(modele.mtl)"></a-entity>
\end{minted}

\subsection{Environnement et Arrière-plan}

\begin{minted}{html}
<!-- Arriere-plan avec couleur solide -->
<a-sky color="#6EBFF5"></a-sky>

<!-- Arriere-plan avec image 360 degres -->
<a-sky src="ciel_360.jpg" rotation="0 -130 0"></a-sky>

<!-- Brouillard pour creer de l'atmosphere -->
<a-scene fog="type: linear; color: #AAA; near: 0; far: 30"></a-scene>
\end{minted}

\subsection{Conseils et Astuces}

\begin{itemize}
    \item \textbf{Système de coordonnées} : Les coordonnées dans A-Frame sont en mètres. L'axe Y est la hauteur (vertical), l'axe X est la largeur (horizontal) et l'axe Z détermine la profondeur (valeurs négatives = éloignement).
    
    \item \textbf{Performances} : Pour une expérience VR fluide, limitez le nombre d'objets complexes et de lumières.
    
    \item \textbf{Test sur mobile} : Pour tester sur smartphone, hébergez votre page sur un serveur web local ou utilisez un service comme Glitch.com.
    
    \item \textbf{Organisation du code} : Utilisez des \texttt{<a-entity>} avec des \texttt{id} pour regrouper des éléments et simplifier leur manipulation.
    
    \item \textbf{Attributs multiples} : Vous pouvez combiner plusieurs attributs sur un même élément.

    \item \textbf{Inspection} : Appuyez sur \texttt{Ctrl+Alt+I} dans une scène A-Frame pour activer l'inspecteur visuel.
\end{itemize}

\subsection{Exemples de Mini-Projets}

\begin{enumerate}
    \item \textbf{Galerie d'images} : Placez plusieurs \texttt{<a-image>} dans un environnement pour créer une exposition.
    
    \item \textbf{Système solaire} : Créez des planètes avec des \texttt{<a-sphere>} et animez-les autour d'un soleil.
    
    \item \textbf{Jeu simple} : Utilisez le curseur pour "attraper" ou faire disparaître des objets.
    
    \item \textbf{Maison virtuelle} : Combinez des \texttt{<a-box>} et autres primitives pour construire une architecture.
\end{enumerate}

\subsection{Ressources Complémentaires}

\begin{itemize}
    \item \textbf{Documentation officielle} : \url{https://aframe.io/docs/}
    \item \textbf{Exemples A-Frame} : \url{https://aframe.io/examples/}
    \item \textbf{Bibliothèque de composants} : \url{https://www.npmjs.com/search?q=aframe-component}
    \item \textbf{Modèles 3D gratuits} : Sketchfab, Google Poly, TurboSquid
    \item \textbf{Textures} : Textures.com, Freepik
    \item \textbf{Éditeur en ligne} : \url{https://glitch.com/~aframe}
\end{itemize}

\end{document}
\documentclass[12pt]{article}
\usepackage[french]{babel}
\usepackage[utf8]{inputenc}
\usepackage[T1]{fontenc}
\usepackage{geometry}
\geometry{a4paper, margin=1in}
\usepackage{hyperref}
\usepackage{xcolor}
\usepackage{titlesec}
\usepackage{enumitem}
\usepackage{tcolorbox}
\tcbuselibrary{skins,breakable}
\usepackage[outputdir=build]{minted}
\usepackage{amssymb}
\usemintedstyle{friendly}

% --- Thème visuel léger et pédagogique (version atelier 30 min) ---
\definecolor{theoblue}{RGB}{23,113,181}
\definecolor{lightblue}{RGB}{239,247,254}
\definecolor{codebg}{RGB}{250,250,250}
\definecolor{coderule}{RGB}{200,200,200}

\titleformat{\section}{\Large\bfseries\color{theoblue}}{\thesection}{0.5em}{}
\titleformat{\subsection}{\large\bfseries\color{theoblue}}{\thesubsection}{0.5em}{}
\setlist[itemize]{topsep=4pt,itemsep=4pt}
\setlist[enumerate]{topsep=4pt,itemsep=4pt}
\setminted{fontsize=\small,breaklines=true,linenos=false,autogobble=true,encoding=utf8,frame=single,bgcolor=codebg,rulecolor=coderule}

\newtcolorbox{objectifbox}{enhanced,breakable,colback=lightblue,colframe=theoblue!70!black,
  boxrule=0.6pt,arc=1mm,left=4mm,right=4mm,top=2mm,bottom=2mm}
\newtcolorbox{checkpoint}{enhanced,breakable,colback=white,colframe=theoblue!50!black,
  boxrule=0.5pt,arc=1mm,left=4mm,right=4mm,top=1mm,bottom=1mm,title=\textbf{\checkmark\;Checkpoint}}
\newtcolorbox{tip}{enhanced,breakable,colback=white,colframe=theoblue!30!black,
  boxrule=0.5pt,arc=1mm,left=4mm,right=4mm,top=1mm,bottom=1mm,title=\textbf{Astuce}}
\newtcolorbox{helpbox}{enhanced,breakable,colback=white,colframe=red!50!black,
  boxrule=0.5pt,arc=1mm,left=4mm,right=4mm,top=1mm,bottom=1mm,title=\textbf{Dépannage rapide}}

\title{Atelier Découverte de la VR Interactive}
\author{Théo AVRIL}
\date{\today}

\begin{document}
\maketitle

\begin{objectifbox}
\textbf{Durée : 30 minutes.}\\
Tu vas : (1) créer une \textbf{page web}, (2) ajouter une \textbf{scène 3D} avec \textbf{A-Frame}, (3) rendre un cube \textbf{interactif} (il change de couleur au clic).\\[2pt]
\textit{Objectif final : voir et manipuler une petite scène 3D dans ton navigateur.}\\[4pt]
\textbf{A-Frame en 2 phrases :} c’est une bibliothèque JS qui permet d’écrire de la \textbf{3D en HTML}. Chaque objet de la scène est une balise : \texttt{<a-box>} (cube), \texttt{<a-sphere>} (sphère), etc.
\end{objectifbox}

\section*{Matériel}
Un \textbf{navigateur moderne} (Chrome/Firefox) et un \textbf{éditeur de texte} (VS Code, etc.).

\section{Étape 1 — Lance ta page web (5 min)}
Crée un fichier \texttt{index.html}, et colle le code suivant :

\begin{minted}{html}
<!DOCTYPE html>
<html>
  <head>
    <meta charset="utf-8">
    <title>Atelier VR Interactif</title>
    <!-- A-Frame -->
    <script src="https://aframe.io/releases/1.2.0/aframe.min.js"></script>
  </head>
  <body>
    <!-- La scène VR sera ajoutée ici -->
  </body>
</html>
\end{minted}

\begin{checkpoint}
Ouvre \texttt{index.html} dans ton navigateur. Si la page s'affiche (même blanche), c'est bon.
\end{checkpoint}

\section{Étape 2 — Ajoute une scène 3D (10 min)}
\textbf{Colle le bloc ci-dessous \emph{à l'intérieur} de la balise \texttt{<body>} (ne duplique pas \texttt{<body>}).}

\begin{minted}{html}
<a-scene>
  <!-- Cube interactif (on ajoutera le script ensuite) -->
  <a-box id="box1" position="-1 0.5 -3" rotation="0 45 0" color="#4CC3D9"
         animation="property: rotation; to: 0 405 0; loop: true; dur: 4000"
         change-color-on-click></a-box>

  <!-- Autres objets -->
  <a-sphere position="0 1.25 -5" radius="1.25" color="#EF2D5E"></a-sphere>
  <a-cylinder position="1 0.75 -3" radius="0.5" height="1.5" color="#FFC65D"></a-cylinder>

  <!-- Sol et ciel -->
  <a-plane position="0 0 -4" rotation="-90 0 0" width="4" height="4" color="#7BC8A4"></a-plane>
  <a-sky color="#ECECEC"></a-sky>

  <!-- Interaction souris Desktop -->
  <a-entity cursor="rayOrigin: mouse"></a-entity>
</a-scene>
\end{minted}

\begin{tip}
\texttt{<a-scene>} crée la scène VR. Les balises \texttt{<a-box>}, \texttt{<a-sphere>}, \texttt{<a-cylinder>} sont des objets 3D prêts à l'emploi.
\end{tip}

\begin{checkpoint}
Résultat attendu : une \textbf{boîte qui tourne}, une sphère, un cylindre, un sol vert et un ciel gris.
\end{checkpoint}

\section{Étape 3 — Rends ton cube interactif (10 min)}
Colle ce script juste avant \texttt{<a-scene>} dans \texttt{<body>} :

\begin{minted}{html}
<!-- Interactivité : changer la couleur d'un objet au clic -->
<script>
  AFRAME.registerComponent('change-color-on-click', {
    init: function () {
      this.el.addEventListener('click', function () {
        // Couleur hex aléatoire fiable (6 chiffres)
        const n = Math.floor(Math.random() * 16777215);
        const randomColor = '#' + n.toString(16).padStart(6, '0');
        this.setAttribute('color', randomColor);
      });
    }
  });
</script>
\end{minted}

\begin{checkpoint}
Teste : clique sur la boîte. \textbf{Résultat attendu :} la \textbf{couleur change}.
\end{checkpoint}

\section*{Bonus (5 min)}
Déplacer la boîte au clic (valeurs numériques) :

\begin{minted}{html}
this.setAttribute('position', {
  x: parseFloat((Math.random() * 4 - 2).toFixed(2)),
  y: 1,
  z: -3
});
\end{minted}

\section*{Aide rapide}
\begin{itemize}
  \item \texttt{position="x y z"} déplace un objet (\(x\)=gauche/droite, \(y\)=haut/bas, \(z\)=profondeur).
  \item \texttt{rotation="x y z"} oriente un objet.
  \item \texttt{color="\#RRGGBB"} change la couleur.
  \item Inspecteur visuel : \texttt{Ctrl+Alt+I}.
\end{itemize}

\begin{helpbox}
\textbf{Rien ne s'affiche ?} Vérifie l'URL d'A-Frame, sauvegarde le fichier, et rafraîchis la page.\\
\textbf{Pas d'interaction ?} Assure-toi d'avoir ajouté \texttt{cursor="rayOrigin: mouse"} dans la scène et que le script est avant \texttt{</body>}.
\end{helpbox}

\section*{Pour aller plus loin (optionnel)}
Exemples et docs : \url{https://aframe.io/examples/} \;|\; \url{https://aframe.io/docs/} \;|\; Éditeur en ligne : \url{https://glitch.com/~aframe}

\end{document}

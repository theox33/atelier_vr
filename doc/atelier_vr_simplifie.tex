
\documentclass[12pt]{article}
\usepackage[french]{babel}
\usepackage[utf8]{inputenc}
\usepackage[T1]{fontenc}
\usepackage{geometry}
\geometry{a4paper, margin=1in}
\usepackage{hyperref}
\usepackage{xcolor}
\usepackage{titlesec}
\usepackage{enumitem}
\usepackage[outputdir=build]{minted}
\usemintedstyle{friendly}

% --- Visuel léger ---
\definecolor{theoblue}{RGB}{23,113,181}
\definecolor{lightblue}{RGB}{239,247,254}
\definecolor{codebg}{RGB}{250,250,250}
\definecolor{coderule}{RGB}{200,200,200}
\titleformat{\section}{\Large\bfseries\color{theoblue}}{\thesection}{0.5em}{}
\titleformat{\subsection}{\large\bfseries\color{theoblue}}{\thesubsection}{0.5em}{}
\setlist[itemize]{topsep=4pt,itemsep=4pt}
\setlist[enumerate]{topsep=4pt,itemsep=4pt}
\setminted{fontsize=\small,breaklines=true,linenos=false,autogobble=true,encoding=utf8,frame=single,bgcolor=codebg,rulecolor=coderule}

\title{Atelier Découverte de la VR Interactive}
\author{Théo AVRIL}
\date{\today}

\begin{document}
\maketitle

\section*{Ce que tu vas faire (30 minutes)}
\begin{itemize}
  \item Créer une \textbf{page web} minimale.
  \item Ajouter une \textbf{scène 3D} avec \textbf{A-Frame}.
  \item Rendre un cube \textbf{interactif} (il change de couleur au clic).
\end{itemize}

\section*{Matériel}
Un navigateur moderne (Chrome/Firefox) et un éditeur de texte (VS Code, etc.).

\section{Étape 1 — Lance ta page web (5 min)}
Crée un fichier \texttt{index.html}, et colle le code suivant :

\begin{minted}{html}
<!DOCTYPE html>
<html>
  <head>
    <meta charset="utf-8">
    <title>Atelier VR Interactif</title>
    <!-- A-Frame -->
    <script src="https://aframe.io/releases/1.2.0/aframe.min.js"></script>
  </head>
  <body>
    <!-- La scène VR sera ajoutée ici -->
  </body>
</html>
\end{minted}

\textit{Ouvre ce fichier dans ton navigateur : tu as une page web prête à accueillir la 3D.}

\section{Étape 2 — Ajoute une scène 3D (10 min)}
Ajoute le contenu suivant dans la balise \texttt{<body>} :

\begin{minted}{html}
<body>
  <a-scene>
    <!-- Cube interactif (on ajoutera le script ensuite) -->
    <a-box position="-1 0.5 -3" rotation="0 45 0" color="#4CC3D9"
           animation="property: rotation; to: 0 405 0; loop: true; dur: 4000"
           change-color-on-click></a-box>

    <!-- Autres objets -->
    <a-sphere position="0 1.25 -5" radius="1.25" color="#EF2D5E"></a-sphere>
    <a-cylinder position="1 0.75 -3" radius="0.5" height="1.5" color="#FFC65D"></a-cylinder>

    <!-- Sol et ciel -->
    <a-plane position="0 0 -4" rotation="-90 0 0" width="4" height="4" color="#7BC8A4"></a-plane>
    <a-sky color="#ECECEC"></a-sky>
  </a-scene>
</body>
\end{minted}

\textbf{À savoir :} \texttt{<a-scene>} crée la scène VR; \texttt{<a-box>}, \texttt{<a-sphere>}, \texttt{<a-cylinder>} sont des objets 3D prêts à l'emploi.

\section{Étape 3 — Rends ton cube interactif (10 min)}
Colle ce script juste avant \texttt{</body>} :

\begin{minted}{html}
<!-- Interactivité : changer la couleur d'un objet au clic -->
<script>
  AFRAME.registerComponent('change-color-on-click', {
    init: function () {
      this.el.addEventListener('click', function () {
        // Couleur hex aléatoire
        var randomColor = '#' + Math.floor(Math.random() * 16777215).toString(16);
        this.setAttribute('color', randomColor);
      });
    }
  });
</script>
\end{minted}

\textbf{Teste :} clique sur le cube — il tourne et change de couleur.

\section{Bonus (5 min)}
Si tu as fini avant les autres, ajoute une autre interaction : déplacer le cube au clic :

\begin{minted}{html}
this.setAttribute('position', {
  x: (Math.random() * 4 - 2).toFixed(2),
  y: 1,
  z: -3
});
\end{minted}

\section*{Aide rapide}
\begin{itemize}
  \item \texttt{position="x y z"} déplace un objet.
  \item \texttt{rotation="x y z"} oriente un objet.
  \item \texttt{color="\#RRGGBB"} change la couleur.
  \item L'inspecteur A-Frame s'ouvre avec \texttt{Ctrl+Alt+I}.
\end{itemize}

\section*{Pour aller plus loin}
\begin{itemize}
  \item Documentation officielle : \url{https://aframe.io/docs/}
  \item Exemples : \url{https://aframe.io/examples/}
  \item Éditeur en ligne : \url{https://glitch.com/~aframe}
\end{itemize}

\end{document}

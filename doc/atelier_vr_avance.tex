\documentclass[12pt]{article}
\usepackage[french]{babel}
\usepackage[utf8]{inputenc}
\usepackage[T1]{fontenc}
\usepackage{geometry}
\geometry{a4paper, margin=1in}
\usepackage{hyperref}
\usepackage{xcolor}
\usepackage{titlesec}
\usepackage{enumitem}
\usepackage{tcolorbox}
\tcbuselibrary{skins,breakable}
\usepackage[outputdir=build]{minted}
\usemintedstyle{friendly}

% --- Thème sobre pour la référence avancée ---
\definecolor{theoblue}{RGB}{23,113,181}
\definecolor{codebg}{RGB}{250,250,250}
\definecolor{coderule}{RGB}{200,200,200}
\titleformat{\section}{\Large\bfseries\color{theoblue}}{\thesection}{0.5em}{}
\titleformat{\subsection}{\large\bfseries\color{theoblue}}{\thesubsection}{0.5em}{}
\setlist[itemize]{topsep=4pt,itemsep=4pt}
\setlist[enumerate]{topsep=4pt,itemsep=4pt}
\setminted{fontsize=\small,breaklines=true,linenos=false,autogobble=true,encoding=utf8,frame=single,bgcolor=codebg,rulecolor=coderule}

\newtcolorbox{concept}{enhanced,breakable,colback=white,colframe=theoblue!50!black,
  boxrule=0.5pt,arc=1mm,left=4mm,right=4mm,top=1mm,bottom=1mm,title=\textbf{Concept}}
\newtcolorbox{checkpoint}{enhanced,breakable,colback=white,colframe=theoblue!50!black,
  boxrule=0.5pt,arc=1mm,left=4mm,right=4mm,top=1mm,bottom=1mm,title=\textbf{Checkpoint}}

\title{Atelier Découverte de la VR Interactive — Version avancée}
\author{Théo AVRIL}
\date{\today}

\begin{document}
\maketitle

% --- Contexte pédagogique bref ---
\begin{concept}
\textbf{Contexte.} A-Frame permet d’écrire de la \textbf{3D en HTML}. Une scène est faite d’\textit{entités} (\texttt{<a-entity>}) auxquelles on ajoute des \textit{composants} (attributs/comportements).
Objectif de la séance : construire une scène \textbf{interactive} (clics, animations, assets) et comprendre la logique \textbf{ECS}.
\end{concept}

\section{Objectifs}
\begin{itemize}
    \item Construire une scène VR et comprendre l'\textbf{approche entités--composants} d'A-Frame.
    \item Créer des \textbf{composants personnalisés} et gérer les \textbf{événements}.
    \item Utiliser \textbf{textures}, \textbf{lumières}, \textbf{modèles 3D} (GLB/OBJ), \textbf{animations}, \textbf{curseur/caméra}.
    \item Appliquer de bonnes pratiques de \textbf{performance} et \textbf{organisation}.
\end{itemize}

\begin{concept}
\textbf{ECS (Entity-Component System).} En A-Frame, tout est \textit{entité} (\texttt{<a-entity>}) qui reçoit des \textit{composants} (attributs) pour lui donner un comportement (\texttt{position}, \texttt{animation}, composants custom, etc.).
\end{concept}

\begin{concept}
\textbf{Prérequis HTTP local et assets.} Servez la page en \textbf{HTTP local} (VS Code Live Server ou \texttt{python -m http.server}) : le chargement d'images/modèles peut échouer en \texttt{file://} (CORS). Placez vos fichiers (\texttt{.jpg}, \texttt{.glb}) dans le \textbf{même dossier} que \texttt{index.html}.
\end{concept}

\section{Démarrage rapide}
\subsection{Page HTML minimale}
\begin{minted}{html}
<!DOCTYPE html>
<html>
  <head>
    <meta charset="utf-8">
    <title>Atelier VR Interactif</title>
    <!-- Version testée : 1.2.0 (ou supérieure) -->
    <script src="https://aframe.io/releases/1.2.0/aframe.min.js"></script>
  </head>
  <body>
    <!-- La scène VR sera définie ici -->
  </body>
</html>
\end{minted}

\subsection{Scène de base + caméra et curseur (collez \emph{à l'intérieur} de votre \texttt{<body>} existant)}
\begin{minted}{html}
<a-scene>
  <a-entity position="0 1.6 0">
    <a-camera>
      <!-- Curseur au centre : clic souris / tap -->
      <a-cursor></a-cursor>
    </a-camera>
  </a-entity>

  <!-- Primitives -->
  <a-box id="box1" position="-1 0.5 -3" rotation="0 45 0" color="#4CC3D9"
         animation="property: rotation; to: 0 405 0; loop: true; dur: 4000"
         change-color-on-click></a-box>

  <a-sphere position="0 1.25 -5" radius="1.25" color="#EF2D5E"></a-sphere>
  <a-cylinder position="1 0.75 -3" radius="0.5" height="1.5" color="#FFC65D"></a-cylinder>

  <a-plane position="0 0 -4" rotation="-90 0 0" width="4" height="4" color="#7BC8A4"></a-plane>
  <a-sky color="#ECECEC"></a-sky>

  <!-- Interaction souris Desktop plug-and-play -->
  <a-entity cursor="rayOrigin: mouse"></a-entity>
</a-scene>
\end{minted}

\begin{checkpoint}
\textbf{Résultat attendu :} un cube bleu qui tourne, une sphère, un cylindre, un sol vert, un ciel gris.\\
Si la page est vide, vérifiez l’URL A-Frame et regardez la console (F12).
\end{checkpoint}

\subsection{Interactivité : composants personnalisés}
\begin{minted}{html}
<script>
  // Composant 1 : couleur aléatoire au clic (toujours 6 chiffres hex)
  AFRAME.registerComponent('change-color-on-click', {
    schema: { },
    init: function () {
      this.el.addEventListener('click', function () {
        const n = Math.floor(Math.random() * 16777215);
        const randomColor = '#' + n.toString(16).padStart(6, '0');
        this.setAttribute('color', randomColor);
      });
    }
  });

  // Composant 2 : déplacement aléatoire au clic (valeurs numériques)
  AFRAME.registerComponent('jump-on-click', {
    schema: { y: {type: 'number', default: 1} },
    init: function () {
      this.el.addEventListener('click', () => {
        const rx = parseFloat((Math.random() * 4 - 2).toFixed(2));
        this.el.setAttribute('position', {
          x: rx, y: this.data.y, z: -3
        });
      });
    }
  });
</script>
\end{minted}

\begin{concept}
\textbf{Pourquoi un composant ?} Un composant rend un comportement \textbf{réutilisable} : ajoutez \texttt{change-color-on-click} sur plusieurs objets sans recopier de code.
\end{concept}

\begin{checkpoint}
\textbf{Résultat attendu :} le cube change de couleur au clic, et peut “sauter” si \texttt{jump-on-click} est appliqué.\\
Sinon, vérifiez que le script est bien juste avant \texttt{</body>}.
\end{checkpoint}

\section{Référence A-Frame (sélection utile)}
\subsection{Formes et objets 3D}
\begin{minted}{html}
<a-box> <a-sphere> <a-cylinder> <a-plane> <a-cone> <a-torus>
<a-ring> <a-circle> <a-triangle> <a-sky>
\end{minted}

\subsection{Attributs communs}
\begin{minted}{html}
position="x y z"  rotation="x y z"  scale="x y z"
color="#ff0000"   opacity="0.5"     visible="true/false"
metalness="0.5"   roughness="0.5"   src="image.jpg"
\end{minted}

\subsection{Lumières (rappels essentiels)}
\begin{minted}{html}
<a-light type="ambient" color="#BBB"></a-light>
<a-light type="directional" position="0 1 1"></a-light>
<a-light type="point" position="0 2 0"></a-light>
<a-light type="spot" position="0 2 0" angle="45"></a-light>
\end{minted}

\subsection{Animations}
\begin{minted}{html}
<a-box position="0 1 -3"
       animation="property: rotation; to: 0 360 0; loop: true; dur: 2000"></a-box>
<!-- Easing, delay, dir, loop, from/to, etc. -->
\end{minted}

\subsection{Événements et curseur/souris}
\begin{minted}{html}
<!-- Active un rayon "souris" pour cliquer sans casque -->
<a-entity cursor="rayOrigin: mouse"></a-entity>

<script>
  // S'assurer que les éléments existent avant d'ajouter des listeners
  window.addEventListener('DOMContentLoaded', () => {
    const el = document.querySelector('#box1');
    if (!el) return;
    el.addEventListener('mouseenter', () => el.setAttribute('scale', '1.1 1.1 1.1'));
    el.addEventListener('mouseleave', () => el.setAttribute('scale', '1 1 1'));
  });
</script>
\end{minted}

\subsection{Textures et images}
\begin{minted}{html}
<a-assets>
  <img id="brick" src="brick.jpg">
</a-assets>
<a-box src="#brick" width="2" height="1" depth="1"></a-box>
\end{minted}

\subsection{Chargement de modèles 3D}
\begin{minted}{html}
<a-assets>
  <a-asset-item id="tree" src="modele.glb"></a-asset-item>
</a-assets>
<a-entity gltf-model="#tree" scale="0.5 0.5 0.5" position="0 0 -5"></a-entity>
\end{minted}

\section{Bonnes pratiques et performances}
\begin{itemize}
  \item Limiter les lumières dynamiques et préférer des \textit{materials} simples.
  \item Réduire le poids des modèles (GLB) et textures (JPEG/PNG optimisés).
  \item Grouper les entités et nommer via \texttt{id} pour les manipulations.
  \item Tester avec l'\textbf{inspecteur} (\texttt{Ctrl+Alt+I}) pour ajuster visuellement.
\end{itemize}

\section{Mini-projets guidés}
\subsection*{1. Galerie 3D}
Démarre en dupliquant \texttt{<a-image>} (change \texttt{src} et \texttt{position}); au clic, agrandis puis reviens.
\subsection*{2. Système solaire}
\texttt{<a-sphere>} pour planètes + animations d'orbite; lumière \texttt{point} au centre.
\subsection*{3. Jeu "attrape-la-boîte"}
Au clic, la boîte se déplace au hasard; compte le score dans une \texttt{<a-text>} simple.

\section{Dépannage (FAQ rapide)}
\begin{itemize}
  \item \textbf{Page vide} : vérifier l'URL A-Frame; regarder la console (F12).
  \item \textbf{Pas de clic} : le composant est-il chargé \emph{avant} \texttt{</body>} ? Activer \texttt{cursor="rayOrigin: mouse"}.
  \item \textbf{Modèle non visible} : vérifier chemin \texttt{src}, \texttt{scale} trop petit, ou erreurs CORS; servir en HTTP local.
\end{itemize}

\section{Ressources}
\begin{itemize}
    \item Documentation : \url{https://aframe.io/docs/}
    \item Exemples : \url{https://aframe.io/examples/}
    \item Composants : \url{https://www.npmjs.com/search?q=aframe-component}
    \item Modèles 3D : Sketchfab, TurboSquid
    \item Textures : Textures.com, Freepik
    \item Éditeur en ligne : \url{https://glitch.com/~aframe}
\end{itemize}

\end{document}
